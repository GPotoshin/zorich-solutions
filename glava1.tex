\documentclass{article}
\usepackage[a4paper,left=3cm,right=3cm,top=1cm,bottom=2cm]{geometry}
\usepackage{amsmath}
\usepackage{amssymb}
\usepackage{hyperref}

\setlength{\parindent}{0mm}

\usepackage{fontspec}
\setmainfont{Linux Libertine O}
\usepackage{unicode-math}
\setmathfont{Cambria Math}

\title{
\textit{\small{Георгий Потошин, 2024}}\\
\vspace{0.3ex}
\textit{\huge{Зорич I, глава 1}}\vspace{1ex}
}

\date{\vspace{-10ex}}

\begin{document}
\maketitle

\section{Логическая символика}

\begin{enumerate}
    \item \textbf{Логические высказывания и действительность.}
        \begin{itemize}
            \item \textbf{"Не"} Возьмём гуся и утку. Утверждение \emph{это гусь}
                верно относительно гуся, но не утки. Его отрицание \emph{это не
                гусь} верно относительно утки, но не верно относительно гуся.
                Тогда будет следующая таблица истинности:\par
                \begin{center}
                \begin{tabular}{ |p{2cm}||p{1cm}|p{1cm}| }
                    \hline
                    $A$ = \emph{это гусь} & гусь & утка\\
                    \hline
                    $A$ & 1 & 0\\
                    \hline
                    $\neg A$ & 0 & 1\\
                    \hline
                \end{tabular}
                \end{center}
            \item \textbf{"И"} Пусть теперь утверждение будет \emph{Черный и
                гусь} (\textit{Plectropterus gambensis}). Тогда напишем
                следующую таблицу всевозможных комбинаций:
                \begin{center}
                    \begin{tabular}{|p{1cm}|p{1cm}|p{1cm}|}
                        \hline
                        & утка & гусь\\
                        \hline
                        белый & 0 & 0\\
                        \hline
                        черный & 0 & 1\\
                        \hline
                    \end{tabular}
                \end{center}
            \item \textbf{"Или"} Пусть теперь утверждение будет \emph{Или
                чёрный или гусь}. Тогда будет следующая таблица:\par
                \begin{center}
                    \begin{tabular}{|p{1cm}|p{1cm}|p{1cm}|}
                        \hline
                        & утка & гусь\\
                        \hline
                        белый & 0 & 1\\
                        \hline
                        черный & 1 & 1\\
                        \hline
                    \end{tabular}
                \end{center}
            \item \textbf{"Следует"} Здесь я ограничусь мыслью, что из правды
                (мы надеемся) всегда следует правда, а изо лжи можно получить
                что угодно. Это мотивировано идеей, что правда одна, а лжи
                много и если бы из правды следовала бы 2 противоположных
                утверждения, то правды уже не было бы одной, а на лжи ловят
                часто из-за противоречий. Поэтому такова таблица.
        \end{itemize}
\end{enumerate}

\end{document}
